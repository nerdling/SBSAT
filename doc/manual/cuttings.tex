\subsection{Translating an expression to canonical form}\label{xlate-section}

\begin{quotation}
\textcolor{red}{Page 7: The section on converting to SAT does not
belong here. At this point the user is trying to figure out how to
use SBSAT. Either move it to a later section, probably on its own,
or save it for the detailed sections (I still believe we will have
one user manual when we are done and one documentation manual).}
\end{quotation}
%
%JVF - let's consider saying something about the canonical form before
%this.

In case an expression is written in a format other than one of those
recognized by {\tt sbsat} it is possible to translate that expression
to the canonical form mentioned on Page~\pageref{can} and discussed in
Section~\ref{can-section}.  Two examples demonstrate this here. The
first is inspired by a well-known problem from the field of Artificial
Intelligence involving a simple database query.  Consider the
following database taken from Charles Dodgson's (aka Lewis Carroll)
book \emph{Symbolic Logic}:

{\baselineskip=0pt
\begin{enumerate}
\item Coloured flowers are always scented.
\item I dislike flowers that are not grown in the open air.
\item No flowers grown in the open air are colourless.
\end{enumerate}}

\noindent
The query is in the form of the question: Do I dislike all flowers
that are not scented?  Express the sentences of the database in first
order logic as:

\begin{enumerate}
\item[] $\forall$\emph{w} ( Colored(\emph{w}) $\Rightarrow$ Scented(\emph{w})
)
\item[] $\forall$\emph{x} ( $\neg$OpenAir(\emph{x})$\Rightarrow$ $\neg$Like(\emph{x})
)
\item[] $\forall$\emph{y} ( OpenAir(\emph{y}) $\Rightarrow$ Colored(\emph{y})
)
\end{enumerate}

\noindent
Express the question as: $\forall$\emph{z} ( $\neg$Scented(\emph{z})
$\Rightarrow$$\neg$Like(\emph{z}) )

Figures~\ref{data-1-figure} and~\ref{data-2-figure} show translations
to two of the formats recognized by {\tt sbsat}: namely, CNF and the
canonical form.  Files {\tt flowers.cnf} and {\tt flowers.ite} in the
{\tt .../examples} subdirectory contain electronic versions of those
figures.

\begin{figure}
\begin{verbatim}
   p bdd 4 4
   ;Coloured flowers are always scented.
   ;I dislike flowers that are not grown in the open air.
   ;No flowers grown in the open air are colourless.
   ;Do I dislike all flowers that are not scented?
   ;Where Colored=1, Scented=2, Like=3, Grown in the Open Air=4
   ;The question must be negated and added to the database.
   ;If SBSAT returns 'unsat' then the answer to the question
   ;is 'YES', otherwise the answer is 'NO'.
   *imp(1, 2)
   *imp(-3, -4)
   *imp(3, 1)
   *not(imp(-2, -4))
\end{verbatim}
\caption{Canonical format for solving the database query problem.}\label{data-1-figure}
\vspace*{6mm}
\begin{verbatim}
   p cnf 4 5
   c Coloured flowers are always scented.
   c I dislike flowers that are not grown in the open air.
   c No flowers grown in the open air are colourless.
   c Do I dislike all flowers that are not scented?
   c Where Colored=1, Scented=2, Like=3, Grown in the Open Air=4
   c The question must be negated and added to the database.
   c If SBSAT returns 'unsat' then the answer to the question
   c is 'YES', otherwise the answer is 'NO'.
   -1 2 0
   3 -4 0
   -3 1 0
   -2 0
   4 0
\end{verbatim}
\caption{CNF format for solving the database query problem.}\label{data-2-figure}
\end{figure}

%JVF - a second example goes here - why not the impulse response?
%JVF - make sure everyone knows that {\tt sbsat} means the executable
%      but SBSAT refers to the package
%JVF - make sure everyone knows that in .../examples, ... means the
%      SBSAT root directory
